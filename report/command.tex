% for table with diagonal box
\usepackage{diagbox}

% for writing pusedo-code
\usepackage[algo2e,vlined,ruled]{algorithm2e}

% for code
\usepackage{listings}
\usepackage{color}
\definecolor{dkgreen}{rgb}{0,0.6,0}
\definecolor{gray}{rgb}{0.5,0.5,0.5}
\definecolor{mauve}{rgb}{0.58,0,0.82}
% \lstset{basicstyle=\footnotesize, numbers=left,numberstyle=\tiny, keywordstyle=\color{blue}, frame=single, extendedchars=false, xleftmargin=2em, xrightmargin=2em, aboveskip=1em, showspaces=false, commentstyle=\color{dkgreen}, stringstyle=\color{mauve}}
\lstset{ %
  language=Octave,                % the language of the code
  basicstyle=\scriptsize,           % the size of the fonts that are used for the code
  numbers=left,                   % where to put the line-numbers
  numberstyle=\tiny\color{gray},  % the style that is used for the line-numbers
  stepnumber=1,                   % the step between two line-numbers. If it's 1, each line 
                                  % will be numbered
  numbersep=5pt,                  % how far the line-numbers are from the code
  backgroundcolor=\color{white},      % choose the background color. You must add \usepackage{color}
  showspaces=false,               % show spaces adding particular underscores
  showstringspaces=false,         % underline spaces within strings
  showtabs=false,                 % show tabs within strings adding particular underscores
  frame=single,                   % adds a frame around the code
  rulecolor=\color{black},        % if not set, the frame-color may be changed on line-breaks within not-black text (e.g. commens (green here))
  tabsize=2,                      % sets default tabsize to 2 spaces
  captionpos=b,                   % sets the caption-position to bottom
  breaklines=true,                % sets automatic line breaking
  breakatwhitespace=false,        % sets if automatic breaks should only happen at whitespace
  title=\lstname,                   % show the filename of files included with \lstinputlisting;
                                  % also try caption instead of title
  keywordstyle=\color{blue},          % keyword style
  commentstyle=\color{dkgreen},       % comment style
  stringstyle=\color{mauve},         % string literal style
  escapeinside={\%*}{*)},            % if you want to add LaTeX within your code
  morekeywords={*,...},               % if you want to add more keywords to the set
  xleftmargin=2em,
  xrightmargin=2em,
  aboveskip=1em,
  belowskip=0em
}



% for table
\usepackage{tabularx}
\usepackage{graphicx}
% \usepackage{booktabs}
%  \usepackage{makecell}
 \usepackage{float}
%  \newcommand{\diff}{\,\mathrm{d}}
% \usepackage[margin=1in]{geometry}
% \usepackage{fancyhdr}
% \pagestyle{fancy}
% \usepackage{extarrows}
% \usepackage{breqn}
% % \usepackage[colorlinks,linkcolor=black]{hyperref}
% \usepackage[colorlinks,linkcolor=blue]{hyperref} % 如果想要目录的颜色是蓝色那就用blue
% \newcommand{\N}{\mathbb{N}}
% \newcommand{\Z}{\mathbb{Z}}
% \newcommand{\trans}{^{\mathrm T}}
% \usepackage{amssymb}
% \usepackage[table]{xcolor}
% \usepackage{bm}
% \usepackage{array}
% \usepackage{mathtools}
% \usepackage[english]{babel}
% \usepackage{natbib}
% \usepackage{url}
% \usepackage[utf8x]{inputenc}
% \usepackage{amsmath}
% \graphicspath{{images/}}
% \usepackage{parskip}
% \usepackage{fancyhdr}
% \usepackage{vmargin}
% \usepackage[font={bf, footnotesize}, textfont=md]{caption}
% \usepackage{amsmath,amsthm,amssymb}


% \newenvironment{theorem}[2][Theorem]{\begin{trivlist}
% \item[\hskip \labelsep {\bfseries #1}\hskip \labelsep {\bfseries #2.}]}{\end{trivlist}}
% \newenvironment{lemma}[2][Lemma]{\begin{trivlist}
% \item[\hskip \labelsep {\bfseries #1}\hskip \labelsep {\bfseries #2.}]}{\end{trivlist}}
% \newenvironment{exercise}[2][Exercise]{\begin{trivlist}
% \item[\hskip \labelsep {\bfseries #1}\hskip \labelsep {\bfseries #2.}]}{\end{trivlist}}
% \newenvironment{reflection}[2][Reflection]{\begin{trivlist}
% \item[\hskip \labelsep {\bfseries #1}\hskip \labelsep {\bfseries #2.}]}{\end{trivlist}}
% \newenvironment{proposition}[2][Proposition]{\begin{trivlist}
% \item[\hskip \labelsep {\bfseries #1}\hskip \labelsep {\bfseries #2.}]}{\end{trivlist}}
% \newenvironment{corollary}[2][Corollary]{\begin{trivlist}
% \item[\hskip \labelsep {\bfseries #1}\hskip \labelsep {\bfseries #2.}]}{\end{trivlist}}
% \DeclareMathOperator{\tr}{tr}
% \DeclareMathOperator{\rank}{rank}
% \DeclareMathOperator{\Span}{span}
% \DeclareMathOperator{\row}{row}
% \DeclareMathOperator{\col}{col}
% \DeclareMathOperator{\range}{range}
% \DeclarePairedDelimiterX{\inp}[2]{\langle}{\rangle}{#1, #2}
% \DeclareMathOperator{\Proj}{Proj}
% \DeclareMathOperator{\trace}{trace}
% \newcommand{\Her}{^{\mathrm H}}
% \DeclareMathOperator{\diag}{diag}
% \makeatletter 
%     \newcommand\fcaption{\def\@captype{table}\caption}
% \makeatother
% \setmarginsrb{2.5 cm}{1.5 cm}{2.5 cm}{1.5 cm}{1 cm}{1 cm}{1 cm}{1 cm}


% \makeatletter
% \let\thetitle\@title
% \let\theauthor\@author
% \let\thedate\@date
% \makeatother

% \pagestyle{fancy}
% \fancyhf{}
% \rhead{\theauthor}
% \lhead{\thetitle}
% \cfoot{\thepage}